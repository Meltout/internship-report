\documentclass[11pt,a4paper]{article}

% --- Page layout (margins < 2cm) ---
\usepackage[a4paper,margin=1.5cm]{geometry}

% --- Fonts & micro-typography ---
\usepackage[T1]{fontenc}
\usepackage[utf8]{inputenc}
\usepackage{lmodern}
\usepackage{microtype}

% --- Math, figures, tables ---
\usepackage{amsmath,amssymb}
\usepackage{graphicx}
\usepackage{booktabs}
\usepackage{caption}
\usepackage{float}

% --- Links ---
\usepackage[hidelinks]{hyperref}
\hypersetup{
  pdftitle    = {Portrait Retouching - Internship Report},
  pdfauthor   = {Antani Gospodinov},
  pdfcreator  = {LaTeX with biblatex/biber},
  pdfkeywords = {}
}

% --- Single spacing (default); keep explicit ---
\linespread{1.0}

% --- Warn if less than 30 pages (non-blocking) ---
\usepackage{lastpage}
\makeatletter
\AtEndDocument{
  \ifnum\value{page}<30
    \PackageWarningNoLine{PageCount}{Document has only \thepage\space pages (minimum 30 required)}
  \fi
}
\makeatother

% --- biblatex / biber setup ---
\usepackage{csquotes} % recommended with biblatex
\usepackage[
  backend=biber,
  style=numeric,   % numeric citations
  sorting=none,    % show in citation order
  url=true,
  doi=true,
  isbn=false,
  giveninits=true, % initials for given names
  maxbibnames=99
]{biblatex}
\addbibresource{refs.bib} % your .bib file

\usepackage{todonotes}

% --- Title (optional) ---
\title{Portrait Retouching \textemdash{} Internship Report}
\author{Antani Gospodinov}
\date{\today}

\begin{document}
\maketitle

\tableofcontents
\newpage

\section{Introduction}
Portrait retouching is a delicate process aimed at reducing minor imperfections --- such as pimples, redness, wrinkles, or stray hairs --- while preserving the subject's unique identity and natural character.
It is common for a professional photographer to capture hundreds of portraits that require retouching from a single assignment --- be it a wedding, modelling photoshoot, or a high school yearbook.
The standard way of removing imperfections involves using a healing brush which allows the photographer to replace blemished regions with other parts of the image.
Popular photo-editing software programs like \emph{Adobe Photoshop}, \emph{GIMP}, \emph{DxO PhotoLab} all offer a healing brush, but unfortunately using it just by itself often produces noticeable discontinuities in the finer textures of the skin, like the pores, or general over-smoothing.
\todo{image}
As a consequence, artists usually supplement the healing brush with other techniques such as \emph{frequency separation} to preserve texture and \emph{dodge and burn} to even out the skin.
In total, a high-quality retouch can take an experienced practitioner more than five minutes per image and unlike more creative-minded image alterations, such as \emph{color grading}, it is generally seen as mundane and uninspiring labor.
The United States saw over two million weddings during 2024.
By making a few conservative estimates that on average 10 photos were captured per wedding and each would take two minutes to retouch we arrive at approximately 666,000 man-hours potentially being consumed by the relatively menial task of manual skin retouching. 
After multiplying this number by \$22.75 --- the reported average hourly pay for a photographer in the U.S. --- we see that automated, high-quality portrait retouching can easily be a multi-million dollar industry around the world.
\todo{source}
\par
Automating the skin retouching process comes with an inherent set of challenges like working with high-resolution images, preserving fine details, and loads of subjectivity around what is and isn't an imperfection. 
In this internship we investigate different paradigms for solving the problem --- classical signal processing, physical modelling of the skin, end-to-end supervised models --- and discuss the upsides and limitations of each.
Our novel contribution is an inference strategy which facilitates end-to-end models trained on lower resolution images to generalize better at high resolutions.
\todo{user study results}
\section{State-of-the-art study}

\section{Method}
We believe that any effective automatization of the skin retouching process should meet the following criteria:
\begin{itemize}
    \item Realistic blemish removal --- ideally we want the end result to look as if the blemish never existed.
    \item Minimal changes in blemish-free regions --- to preserve the subject's identity as much as possible.
    \item Fast inference on high-resolution photos --- up to 20 megapixels and more.
\end{itemize}

\subsection{Inference Strategies}

\subsection{Fine-Tuning}

\section{Results}

\section{Discussion}

\section{Conclusion}

% ===== References =====
\printbibliography

\end{document}